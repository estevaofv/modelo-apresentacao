% -*- coding: utf-8 -*-
% -*- mode: latex -*-
%
% Template NExT Beamer Theme
%
% Autor       : Pedro Maione [pedromaionee@gmail.com]
% Copyright   : Copyright(c) 2017 NExT. Todos os direitos reservados.
% Descrição   : Modelo de apresentação do Beamer com tema do NExT.
% Versão      : 2017-02
%

\documentclass[10pt,aspectratio=43,xcolor,compress]{beamer}
% --------------------------------------------------------------------------
% Pacotes e configurações
% --------------------------------------------------------------------------
\usepackage{lipsum}
\usepackage{tabularx,ragged2e}
\usepackage{booktabs}



% Insere arquivo com configurações para listar códigos

% Configurações do pacote `listings'

\usepackage{listings}
\lstset{literate=
  {ã}{{\~a}}1 {à}{{\`a}}1 {À}{{\`A}}1
  {á}{{\'a}}1 {é}{{\'e}}1 {í}{{\'i}}1 {ó}{{\'o}}1 {ú}{{\'u}}1
  {Á}{{\'A}}1 {É}{{\'E}}1 {Í}{{\'I}}1 {Ó}{{\'O}}1 {Ú}{{\'U}}1
  {â}{{\^a}}1 {ê}{{\^e}}1 {î}{{\^i}}1 {ô}{{\^o}}1 {û}{{\^u}}1
  {Â}{{\^A}}1 {Ê}{{\^E}}1 {Î}{{\^I}}1 {Ô}{{\^O}}1 {Û}{{\^U}}1
  {ç}{{\c c}}1 {Ç}{{\c C}}1
}

\lstdefinestyle{customlatex}{ %
  language=[LaTeX]TeX,
  columns=fullflexible,
  breaklines=true,
  breakatwhitespace=true,
  backgroundcolor=\color{corCinzaClaro!15},
  basicstyle=\ttfamily\footnotesize,
  commentstyle=\color{corSecund3Escuro},
  keywordstyle=\bfseries\color{corSecund2},
  stringstyle=\color{corSecund3Escuro},
  showspaces=false,
  showstringspaces=false,
  showtabs=false,
  keepspaces=true,
  tabsize=4,
  escapeinside='',
}

\lstset{ %
  style=customlatex,
  numbers=left,
  numberstyle=\scriptsize\sffamily,
  stepnumber=1,
  frame=tb,
  framerule=0.5pt,
  framexleftmargin=0.5em,
  framexrightmargin=0.5em,
  xleftmargin=0.5em,
  xrightmargin=0.5em
}

%%% Local Variables:
%%% coding: utf-8
%%% mode: latex
%%% TeX-master: "apresentacao-next-demo"
%%% End:


\hypersetup{%
  linkcolor=corPrimariaEscuro,
  urlcolor=magenta,
  colorlinks=true,
}

% --------------------------------------------------------------------------
% Carrega o Tema
% --------------------------------------------------------------------------
\usetheme{next}

% --------------------------------------------------------------------------
% Configurações da apresentação
% --------------------------------------------------------------------------
\title{Título da Apresentação}
\subtitle{Subtitulo da apresentação}
\date{\today}
\author{Pedro Maione}
\institute{\NExT}

% --------------------------------------------------------------------------
% Comandos personalizados
% --------------------------------------------------------------------------
\newcommand{\NExT}{Núcleo de Estudos e Pesquisas Experimentais e Tecnológicas}

% --------------------------------------------------------------------------
% Configuração de anotações
% --------------------------------------------------------------------------
\setbeameroption{show notes}

\begin{document}
% --------------------------------------------------------------------------
% Slide com Título
% --------------------------------------------------------------------------

\maketitle

% --------------------------------------------------------------------------
% Sumário
% --------------------------------------------------------------------------
\section*{Sumário}
\begin{frame}{Sumário}
  \tableofcontents[hideallsubsections]
\end{frame}

% --------------------------------------------------------------------------
% Conteúdo
% --------------------------------------------------------------------------
\section{Introdução}

\begin{frame}{Introdução}

  Este é o modelo de apresentação do \textbf{\NExT} (\textbf{NExT}).
  Ele está disponivel em \url{http://nextpesquisa.com.br/downloads}

\end{frame}

\section{Utilização}

\begin{frame}[containsverbatim]
  \frametitle{Frames}

  Para inserir slides, utilize o comando \lstinline|frame|.

\begin{lstlisting}[style=customlatex]
\begin{frame}{Titulo}
  % Comentário
  '\textless{} \emph{Seu código aqui} \textgreater'
\end{frame}
\end{lstlisting}

  O título do \emph{slide} pode ser inserido como no exemplo anterior ou utilizando o comando \lstinline|frametitle|.

\begin{lstlisting}[style=customlatex]
\begin{frame}
  \frametitle{Titulo}
  % Comentário
  '\textless{} \emph{Seu código aqui} \textgreater'
\end{frame}
\end{lstlisting}

\end{frame}

\begin{frame}
  \frametitle{Blocos}

  \begin{block}{Título do Bloco}
    Este modelo foi preparado como uma aplicação para apresentações do \textbf{NExT}.
  \end{block}

  \begin{exampleblock}{Título do Bloco de Exemplo}
    Este é um exemplo de bloco de exemplo.
  \end{exampleblock}

  \begin{alertblock}{Título do Bloco de Alerta}
    \textbf{Atenção!!!}
  \end{alertblock}


\end{frame}

\begin{frame}{Referências}

  \begin{block}{CTAN}
    The \emph{Com­pre­hen­sive TeX Archive Net­work} (CTAN) é o lugar que centraliza todo tipo de material sobre \TeX{}. 
  \end{block}

  \begin{itemize}
  \item Mais informações sobre os pacotes do \LaTeX{} podem ser encontradas em \url{http://www.ctan.org/};
  \item Para maiores informações a respeito do Beamer, consulte o guia do usuário em \url{https://www.ctan.org/pkg/beamer};
  \end{itemize}

  \vfill{}

  Outras fontes úteis:
  \begin{enumerate}
  \item \url{http://www.latex-project.org/}
  \item \url{http://tex.stackexchange.com/}
  \item \url{http://www.tug.org/}
  \end{enumerate}

\end{frame}


\newcommand{\ExemploPaletaCor}[2]{%
  \setbeamercolor{exemploPaleta}{fg=#1,bg=#2}
  \begin{beamercolorbox}[wd=\linewidth,ht=2ex,leftskip=.5ex,dp=0.7ex]{exemploPaleta}
    \texttt{#1}
  \end{beamercolorbox}
}

\newcommand{\ExemploPaletaCorBg}[2]{%
  \setbeamercolor{exemploPaleta}{fg=#2,bg=#1}
  \begin{beamercolorbox}[wd=\linewidth,ht=2ex,leftskip=.5ex,dp=0.7ex]{exemploPaleta}
    \texttt{#1}
  \end{beamercolorbox}
}

\section{Referências}

\begin{frame}{Paleta de cores}
  \begin{multicols}{2}
    \ExemploPaletaCor{corPrimariaClaro}{white}
    \ExemploPaletaCor{corPrimaria}{white}
    \ExemploPaletaCor{corPrimariaEscuro}{white}
    %
    \ExemploPaletaCorBg{corPrimariaClaro}{white}
    \ExemploPaletaCorBg{corPrimaria}{white}
    \ExemploPaletaCorBg{corPrimariaEscuro}{white}
  \end{multicols}

  \begin{multicols}{2}
    \ExemploPaletaCor{corSecund1Claro}{white}
    \ExemploPaletaCor{corSecund1}{white}
    \ExemploPaletaCor{corSecund1Escuro}{white}
    %
    \ExemploPaletaCorBg{corSecund1Claro}{white}
    \ExemploPaletaCorBg{corSecund1}{white}
    \ExemploPaletaCorBg{corSecund1Escuro}{white}
  \end{multicols}

  \begin{multicols}{2}
    \ExemploPaletaCor{corSecund2Claro}{white}
    \ExemploPaletaCor{corSecund2}{white}
    \ExemploPaletaCor{corSecund2Escuro}{white}
    %
    \ExemploPaletaCorBg{corSecund2Claro}{white}
    \ExemploPaletaCorBg{corSecund2}{white}
    \ExemploPaletaCorBg{corSecund2Escuro}{white}
  \end{multicols}

    \begin{multicols}{2}
    \ExemploPaletaCor{corSecund3Claro}{white}
    \ExemploPaletaCor{corSecund3}{white}
    \ExemploPaletaCor{corSecund3Escuro}{white}
    %
    \ExemploPaletaCorBg{corSecund3Claro}{white}
    \ExemploPaletaCorBg{corSecund3}{white}
    \ExemploPaletaCorBg{corSecund3Escuro}{white}
  \end{multicols}

\end{frame}

\end{document}
